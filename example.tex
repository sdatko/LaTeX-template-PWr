\documentclass[4:3,pl,v2]{template-PWr/presentation}

%
% Metadata
%
\title{Tytuł}
\subtitle{Podtytuł}
\author{Autor}
\contact{Kontakt}
\institute{Jednostka}
\date{\today}
\keywords{Słowa kluczowe; Oddzielone średnikami}
\subject{Krótkie omówienie zawartości}


%
% Content
%
\begin{document}
    \begin{frame}
        \setstretch{1.5}

        \titlepage{}
    \end{frame}


    \begin{frame}
        \frametitle{Przykładowa treść}
        \setstretch{1.25}
        \footnotesize

        \begin{itemize}
            \setlength{\itemsep}{0.5em}
            \item Zawartość domyślnie jest wyśrodkowana w pionie.
            \item Listy dobrze się sprawdzają przy definiowaniu treści.
            \item Zagnieżdżanie list jest jak najbardziej możliwe.
                  \begin{itemize}
                      \item[--] A nawet na więcej niż jeden poziom!
                                \begin{itemize}
                                    \item[-] Można określić własny styl wypunktowania.
                                \end{itemize}
                  \end{itemize}
        \end{itemize}

        \vfill

        \begin{enumerate}
            \setlength{\itemsep}{0.5em}
            \item Zawartość domyślnie jest wyśrodkowana w pionie.
            \item Listy dobrze się sprawdzają przy definiowaniu treści.
            \item Zagnieżdżanie list jest jak najbardziej możliwe.
                  \begin{enumerate}
                      \item A nawet na więcej niż jeden poziom!
                            \begin{enumerate}
                                \item Można określić własny styl wypunktowania.
                            \end{enumerate}
                  \end{enumerate}
        \end{enumerate}
    \end{frame}
\end{document}
